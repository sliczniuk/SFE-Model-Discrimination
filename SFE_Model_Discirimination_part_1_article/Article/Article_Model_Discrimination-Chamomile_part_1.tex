% ---------------------------------------------------------------
% Preamble
% ---------------------------------------------------------------
%\documentclass[a4paper,fleqn,longmktitle]{cas-sc}
\documentclass[a4paper,fleqn]{cas-dc}
%\documentclass[a4paper]{cas-dc}
%\documentclass[a4paper]{cas-sc}
% ---------------------------------------------------------------
% Make margins bigger to fit annotations. Use 1, 2 and 3. TO be removed later
%\paperwidth=\dimexpr \paperwidth + 6cm\relax
%\oddsidemargin=\dimexpr\oddsidemargin + 3cm\relax
%\evensidemargin=\dimexpr\evensidemargin + 3cm\relax
%\marginparwidth=\dimexpr \marginparwidth + 3cm\relax
% -------------------------------------------------------------------- 
% Packages
% --------------------------------------------------------------------
% Figure packages
\usepackage{graphicx,float}
\usepackage{adjustbox}
% Text, input, formatting, and language-related packages
\usepackage[T1]{fontenc}
\usepackage{subcaption}
\usepackage{paralist}
%\usepackage{csvsimple}

% TODO package
\usepackage[bordercolor=gray!20,backgroundcolor=blue!10,linecolor=black,textsize=footnotesize,textwidth=1in]{todonotes}
\setlength{\marginparwidth}{1in}
% \usepackage[utf8]{inputenc}
% \usepackage[nomath]{lmodern}

% Margin and formatting specifications
%\usepackage[authoryear]{natbib}
\usepackage[sort]{natbib}
\setcitestyle{square,numbers}

 %\bibliographystyle{cas-model2-names}

\usepackage{setspace}
\usepackage{subfiles} % Best loaded last in the preamble

% \usepackage[authoryear,longnamesfirst]{natbib}

% Math packages
\usepackage{amsmath, amsthm, amssymb, amsfonts, bm, nccmath, mathdots, mathtools, nccmath, bigints, ulem}

\usepackage{tikz}
\usepackage{pgfplots}
\usetikzlibrary{shapes.geometric,angles,quotes,calc}

\usepackage{placeins}

\usepackage[final]{pdfpages}

% --------------------------------------------------------------------
% Packages Configurations
\usepackage{enumitem}
% --------------------------------------------------------------------
% (General) General configurations and fixes
\AtBeginDocument{\setlength{\FullWidth}{\textwidth}}	% Solves els-cas caption positioning issue
\setlength{\parindent}{20pt}
%\doublespacing
% --------------------------------------------------------------------
% Other Definitions
% --------------------------------------------------------------------
\graphicspath{{Figures/}}
% --------------------------------------------------------------------
% Environments
% --------------------------------------------------------------------
% ...

% --------------------------------------------------------------------
% Commands
% --------------------------------------------------------------------

% ==============================================================
% ========================== DOCUMENT ==========================
% ==============================================================
\begin{document} 
%  --------------------------------------------------------------------

% ===================================================
% METADATA
% ===================================================
\title[mode=title]{Design of experiment for model discrimination: SFE case, part I}                      
\shorttitle{ODOE-DM}

\shortauthors{OS}

\author[1]{Oliwer Sliczniuk}[orcid=0000-0003-2593-5956]
\ead{oliwer.sliczniuk@aalto.fi}
\cormark[1]
\credit{a}

%\author[1]{Pekka Oinas}[orcid=0000-0002-0183-5558]
%\credit{b}

%\author[1]{Francesco Corona}[orcid=0000-0002-3615-1359]
%\credit{c}

\address[1]{Aalto University, School of Chemical Engineering, Espoo, 02150, Finland}
%\address[2]{2}

\cortext[cor1]{Corresponding author}

% ===================================================
% ABSTRACT
% ===================================================
\begin{abstract}
This study investigates the process of chamomile oil extraction from flowers. A parameter-distributed model consisting of a set of partial differential equations is used to describe the governing mass transfer phenomena in a solid-fluid environment under supercritical conditions using carbon dioxide as a solvent. The concept of quasi-one-dimensional flow is applied to reduce the number of spatial dimensions. The flow is assumed to be uniform across any cross-section, although the area available for the fluid phase can vary along the extractor. The physical properties of the solvent are estimated from the Peng-Robinson equation of state. Based on the set of laboratory experiments performed under multiple constant operating conditions: $30 - 40^\circ C$, $100 - 200$ bar, and $3.33-6.67 \cdot 10^{-5}$ kg/s, two process models are developed. The goal of this work is to design an experiment to discriminate between two competing process models. Statistical methods, such as the Kolmogorov-Smirnov and Mann–Whitney U test, and Jensen–Shannon divergence, are used to discriminate between two models under the parameter uncertainty.

\end{abstract}

\begin{keywords}
Supercritical extraction \sep Optimal design of experiment \sep Model discrimination \sep Mathematical modelling
\end{keywords}

% ===================================================
% TITLE
% ===================================================
\maketitle

% ===================================================
% Section: Introduction
% ===================================================

\section{Introduction}

\subfile{Sections/introduction_imp}

\subfile{Sections/Literature_Review}

% ===================================================
% Section: Main
% ===================================================

\subfile{Sections/Model}

%\subfile{Sections/RBF}

\subfile{Sections/ODoE_MD}

% ===================================================
% Section: Summary
% ===================================================

%\section{Results}
%\subfile{Sections/Results_DOE}

\section{Conclusions} \label{CH: Conclusion}

This work focus on model discrimination, which is a part of model development process. After several challenger models are developed, the champion model needs to be selected based on case-specific criteria. These might include model accuracy, complexity or predictive power. The predictive power can be tested on validation dataset, which it self can be a random subset of the development dataset or a new experiment to be yet performed. In the second case, it is beneficial to select such operating conditions that both models differ the most. Such design of an experiment 


% ===================================================
% Bibliography
% ===================================================
%% Loading bibliography style file
\newpage
%\bibliographystyle{model1-num-names}
\bibliographystyle{unsrtnat}
\bibliography{mybibfile}

\clearpage \appendix \label{appendix}
\section{Appendix} 
\subsection{Parameter estimation of the challenger model}
\subfile{Sections/App_3_Challanger_Model} \label{chap:Parameter_Estimation_Challanger}

\onecolumn
\subsection{Temporal evolution of output output distributions}
\begin{figure*}[!b]
	\begin{subfigure}[b]{0.45\textwidth}
		\centering
		\includegraphics[trim = 2.0cm 3.0cm 15.0cm 2.0cm,clip,width=\columnwidth]{Figures/Results/distribution_P200_T30_F333.png}
		\caption{F=3.33e-5 kg/s}
		\label{fig:Yield_low_F_evolution}
	\end{subfigure}
	\hfill
	\begin{subfigure}[b]{0.45\textwidth}
		\centering
		\includegraphics[trim = 2.0cm 3.0cm 15.0cm 2.0cm,clip,width=\columnwidth]{Figures/Results/distribution_P200_T30_F667.png}
		\caption{F=6.67e-5 kg/s}
		\label{fig:Yield_high_F_evolution}
	\end{subfigure}
	\caption{Temporal evolution of cumulative distribution at T=30$^\circ$ C, P=200 bar}
	\label{fig:Yield_dist_evolution}
\end{figure*} 

\begin{figure*}[h]
	\begin{subfigure}[b]{0.45\textwidth}
		\centering
		\includegraphics[trim = 3.0cm 3.0cm 15.0cm 2.0cm,clip,width=\columnwidth]{Figures/Results/distribution_rate_P200_T30_F333.png}
		\caption{F=3.33e-5 kg/s}
		\label{fig:Rate_low_F_evolution}
	\end{subfigure}
	\hfill
	\begin{subfigure}[b]{0.45\textwidth}
		\centering
		\includegraphics[trim = 3.0cm 3.0cm 15.0cm 2.0cm,clip,width=\columnwidth]{Figures/Results/distribution_rate_P200_T30_F667.png}
		\caption{F=6.67e-5 kg/s}
		\label{fig:Rate_high_F_evolution}
	\end{subfigure}
	\caption{Temporal evolution of extraction rate distribution at T=30$^\circ$ C, P=200 bar}
	\label{fig:Rate_dist_evolution}
\end{figure*} 

%\subsection{Cardano's Formula} \label{CH: Cardano}
%\subfile{Sections/Cardano}

\end{document}